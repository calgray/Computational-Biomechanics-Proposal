

\begin{comment}
Overview:

The project proposal needs to demonstrate that a student is adequately prepared to take on research work, leading to a thesis.  The marker will be looking to see sound answers to the following questions:
Is there sufficient background information to justify the specific topic of the proposal and to give a sense of a bigger picture?
Is a problem clearly identified and stated?
Is the student’s intended contribution towards solving the problem clearly stated?
Is the review of previous work and/or the literature sufficiently extensive at this stage (within expectations of the particular project) without losing focus on the topic of the proposal?
Is the review well structured?
Are the sources cited in the review appropriate?
Are the majority of the sources scholarly and/or professional (journals, technical papers/publications, industry data – not Wikipedia or similar websites)?
Is there a sufficiently detailed explanation of how the research will be carried out and what has already been accomplished?
Have all necessary and appropriate resources been identified and justified?
Is the timeline for completion of the research realistic and is it sufficiently detailed?
The quality of the writing is expected to be of a high standard, showing both a command of English (spelling, punctuation and vocabulary) and an understanding of the technical language appropriate to the field.
All material appearing in the proposal that is not the student’s original work must be appropriately referenced and a consistent citation style must be used throughout the document.
The proposal must follow the 2000-3000 word guidelines (unless otherwise specified) and be free of filler (e.g. numerous block quotes to pad out the word count).
The Project Proposal contributes 15% to the student’s final mark for the Engineering Research Project.

\end{comment}