% Introduction or Background
% This provides the reader with the context of the project. For example, what is the application area, why is it important, what (in general terms) has been done before?

% Marking
% The proposal makes clear the context in which the work is taking place. It is clear to the reader why the problem/proposed question is important or relevant within the field of study.
% Mark/10, Weight = 2

%This may bit a little tricky to keep seperate from Literature Review/Previous work, maybe do just an intro about what the overall aim of the series of projects relating to multirotors is all for.


%Meshless algorithms consist of techniques for processing 3D volumes using approaches other than non-conventional 3D surface models. This paper will discuss an techniques for deformation and volume rendering of 3D dimensional cloud of points data that involves the simulation of biological materials in conjunction with algorithms for 3D surface construction and presentation suitable for real-time surgical simulation. While very soft tissue dynamics have been simulated for finite element methods there exists no post and pro-processors for meshless algorithms.

%This proposal will cover a meshless algorithm pre and post-processor using Total Lagrangian Explicit Dynamics.


Surgical intervention of common and uncommon procedures require several hard to obtain resources including trained surgeons, CT and MRI scans, personnel, appropriate operating infrastructure, scanners, anesthetics, tools and time for intervention planning in order to produce successful outcomes. For emergency and remote circumstance, not all of these resources can be adequately met for a high chance of surgical success but could otherwise be provided by alternative approaches in the area of preparation and planning.

One approach for achieving better operating success rates is via the availability and utilizing of effective simulation and planning tools whereby a surgeon can both quickly and conveniently create and test a plan for surgical intervention with a higher degree of confidence. Under regular circumstances, surgical planning and conventional approach utilizes CT and MRI scans of the patient and is processed by a doctor often in the form of 2D cross sectional images to identify regions of operation and plan an appropriate method of intervention. The challenge with this approach is performing the surgical intervention quickly and appropriately in emergency and remote circumstances is difficult where time or medical experience is limited.

The proposed solution in this paper will iterate on the open source platform tools slicer 3D as image-guided interventions whereby a new planning processes will readily take advantage of visualization, data import/export, and image processing algorithms \cite{Ungi2016} provided by the platform. This proposal will discuss an alternative means of displaying 3D MRI and CT data to a user for the purpose of surgical intervention planning utilizing both meshless algorithms of surgical simulation and virtual reality by providing a surgical intervention planning tool that can allow for both fast recording of an intervention plan followed by a simulation displaying the procedure to in order to establish a higher degree of confidence in performing the operation on the patient.


%Meshless algorithms consist of techniques for processing 3D volumes using approaches with higher accuracy than finite element methods. This paper will discuss techniques for simulation of soft body materials in conjunction with 3D surface construction suitable for real-time surgical simulation.
