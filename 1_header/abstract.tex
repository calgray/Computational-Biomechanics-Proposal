\begin{comment}
Background Information
Creates a clear picture of the field

Problem Statement
The problem is stated clearly and specifically
Intended Contribution
The intended contribution is stated clearly and specifically

Extent of Review
The review is extensive and focused

Structure of Review
The structure is logical and leads to the identified problem and proposed method of research

Appropriate Sources
The sources are scholarly and or professional

Explanation of Method
It is clear how the research will be carried out

Identification of Resources
All necessary resources have been identified and their requirement justified 

Timeline to Completion
The timeline is realistic and detailed

Quality of Writing 
The writing is excellent

Citations
All sources have been cited

Appropriate Length 
The length is appropriate (body of document 2000-3000 words)
\end{comment}

Surgical simulation is an area of computational biomechanics with the potential to provide valuable intervention training and planning resources to modern medicine. With the aid of a meshless total Lagrangian explicit dynamics (MTLED) framework, it is possible to perform surgical simulation utilizing patient specific models created directly from radiographic images with accuracy and satisfying the requirements of image-guided surgery. Creation of useful surgical simulation software in conjunction with simulation requires efficient methods of human interfacing to perform pre-processing, planning, simulated operation, and post-processing of simulated results by medical professionals. This proposal will discuss new approaches for preparing collected 3D data via a virtual reality human interface device to provide accurate simulation and easy to interpret planning and testing techniques for surgical interventions.