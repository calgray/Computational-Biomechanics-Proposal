\begin{comment}
Compiler Suite
* CMake
* Visual Studio 15 2017 Win64
* CMake 3.17.0-rc2
* Qt 5.13.0
\end{comment}

%  Methodology and Progress to Date
% How you plan to solve the problem or resolve the hypothesis.  Summarise your progress towards the project’s objectives.  This might include preliminary results. 
% Marking
% The proposal presents a reasonable plan to address the problem given the student's experience in the area of study.
% Mark/10, Weight = 2

\label{section:methods}

The method for solving the challenges in this project can be divided into 3 areas:

\subsection{Input}
The current virtual reality extensions of the 3D Slicer software platform consist of a virtual reality module that utilizes head tracking for a user to transform themself around a Slicer scene in addition to viewing a stereoscopic perspective of models and slices of these at the camera near clipping plane. The proposed input component of this project will consist of additionally providing a means for a user of a virtual reality headset to:

\begin{itemize}
  \item orientate and transform models
  \item place and remove models
  \item resume and pause simulation
  \item control visualization settings (either via virtual sliders or swapping between settings) for: contrast, transparency, overlaying first and last frame, colorization by classification, stress, strain and density.
\end{itemize}

Each of the above items are currently possible to perform using the standard Slicer 3D interface but not available to a user wearing a virtual reality headset. Two of the most popular VR platforms for desktop, Oculus Rift and HTC Vive, provide handheld motion tracking devices consisting of 4 buttons and joystick per hand that allow for more instinctive user interaction via grabbing and pointing approaches as apposed to memorization of key mappings.

For this section the proposed method for input is to provide a virtual workbench where a user can select a variety of virtual tools via grabbing items from the workbench that contain specific functionality that integrates with the virtual reality touch controllers.

\subsubsection{Basic Virtual Tools}
Virtual tools to be made will include following basic tools:
\begin{itemize}
\item Select Tool - special tool for selecting models by raycasting from the users pointing hand tracker to first intersected model surface. Selection is then performed by pressing holding a button on the opposite hand and visual indicator shows the item is selected for use by the other virtual reality tools.
\item Helping Hands Tool - For transforming a selected model into a fixed position and orientation (excludes scaling).
\item Scaling Tool - For scaling one or more model/s to a discrete set of scale ratios.
\item Information Tool - For displaying and modifying properties of a selected model or segmentation.
\end{itemize}

\subsubsection{Advanced Virtual Tools}
To assist in the planning of surgery within Slicer, the following advanced tools will be considered for creation to manipulate models:
\begin{itemize}
\item Segmentation Paint Tool - Tool to paint and erase segment areas. 
\item Cutting Tool - Casts a ray through a segmentation to convert it into two segmentations. This could be useful for interventions that involve extraction or re-positioning of bones and organs. 
\item Animation Tool - Provides controls to plays and animate a planned intervention.
\end{itemize}

\subsection{Visualization}
The visualization component of this project will consist of providing a time series animated view of the simulated material which may additionally be viewed in conjunction with either the first and last frame of simulation using transparency layers.

Slicer 3D uses 3 separate scene hierarchies each for for subjects, their transforms and segmentations. While a regular user with a screen has access and control of these lists, a virtual reality user currently only has control over the model to room transformation.

\subsubsection{Model View and Selection}
This view extension will provide the ability for a user to select which model they wish to work with. This can be done by actively displaying a virtual list of models by name and a visual preview of the model either in the list or in the active scene. 

\subsubsection{Segmentation View and Selection}
Segmentations of a model are shaded in separate colors by default but provides no visual stimulus for selecting segmentation The two proposed approaches for selection in this context is to either display a 3D bounding box of the the currently selected segmentation or to provide a blue outline shading effect to indicate which segmentation is selected (even within a group of superimposed segmentations).

\subsubsection{Animation View}
Animation is available within Slicer 3D but there exists no means of playing, stopping and controlling segmentation visibility within virtual reality. A new view is proposed that will allow for a user to separately play animations for surgical simulation with the possibility of allowing pre-calculated model deformations where possible.

\subsection{Model Pre-Processing}
Model preparation will be required to load data formats of particular patients in addition to exporting and importing time series data of pre-simulated meshless models.

\subsection{Python API}
To assist in handling feedback related changes it will be desirable to support virtual reality configuration through a Python API for controlling aspects such selecting available simulation features, loading demos, automated testing, and configuring parameters that may or may not be possible to configure to user in virtual reality. This will require language interop via the available python headers in C++.

\subsection{Performance Profiling}
Virtual reality performance is a critical area for practicality reasons whereby a minimum frame rate must be achieved on modern systems in order for user experiences to be positive. This project will aim for providing at least 50 frames per seconds to meet acceptable 20ms end-end delay time guidelines \cite{Raaen2018} by measuring average frame rate with a modern high performance GPU and 8 virtual cores for processing on test datasets. This will be performed using the integrated SteamVR GPU profiler.

Any issues that actively prevent a sub 20ms end-end delay time for head mounted displays (HMDs) will be investigated and documented for further evaluation.

