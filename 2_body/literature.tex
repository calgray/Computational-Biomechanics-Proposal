% Literature Review or Previous Work

% Explain the literature (eg refereed research papers) or previous body of work (eg previous projects within the research group) on which your investigation is based. This should not simply be a linear account, but rather a synthesis of what is important from what has gone before. It will often be a hierarchical account, moving from a general understanding of the field, to identification and expansion of work that is specifically relevant to your project. The literature review may be completed as a group, and included as an appendix to the proposal. If completed as a group, all authors should be listed.

%Marking
% Literature Review/Previous Work (may be joint work providing all authors credited)
% The review provides a succinct and well structured account of previous work/theoritical models and framework on which the project will build. (This may include research literature, previous projects, existing artefacts, etc, depending on the type of project.) The review provides more than just a list of previous work/theories and frameworks but draws it together, highlighting relevant work and connections, to provide the reader with an understanding of how the student's work builds on existing work.
% Mark/10 Weight = 5

%\textit{Check 2015 proposal past work, summary of 2015 thesis work, summary of the current state of components and capabilities.}

\begin{comment}


Meshless models for simulating materials

Homogenous Materials:
Brain Deformations: https://www.sciencedirect.com/science/article/pii/S002192900600090X

Meshless Lagrangian explicit dynamics algorithm:
http://isml.ecm.uwa.edu.au/ISML/Publication/pdfs/2010hortonmillerIJfNMiBEmeshless.pdf

Volume rendering consists of techniques for 3D surface construction.

Non linear methods:
Moving least squared: http://www.cs.utah.edu/~csilva/courses/cpsc7960/pdf/lancaster-salkauskas-mls.pdf

Total Lagrangian explicit dynamics finite element algorithm for computing soft tissue deformation, 2006

Patient-specific model of brain deformation: Application to medical image registration, 2007

Open-source platforms for navigated image-guided interventions, 2016

(Hyper)-graphical models in biomedical image analysis, 2016


Camera tracking objects is bad, handling occlusions is problematic


Total Lagrangian explicit dynamics finite element algorithm for computing soft tissue deformation \cite{Miller2007}

Patient-specific model of brain deformation: Application to medical image registration
\cite{Wittek2007}

A meshless Total Lagrangian explicit dynamics algorithm for surgical simulation
\cite{Horton2010}

\end{comment}

\subsection{Surgical Simulation}

As part of ongoing efforts of reducing costs to meet the rising expectations for accessible health care, computer-integrated surgery (CIS) is an area of technological advancement being developed to overcome the limitations of traditional surgery. Surgical simulation is a particular area of CIS aimed to model and simulate deformable materials for applications requiring real-time interaction such as simulation-based training, skills assessment and operation planning.\cite{Miller2007} While many accurate real-time simulations exist for a variety of traditional engineering processes such as mining, transport, construction and control systems, extending the success of computational mechanics to biological fields such as biomedical science and medicine has proven challenging. \cite{Tinsley2003}

\subsection{Reliability of Computational Mechanics}

Reliability in the area of computer-generated predictions is one of the major concerns to specialists in their respective areas. The process in which this is managed is known as verification and validation. In the context of computational mechanics, validation relates to determining the appropriateness of the scientific principles and mathematical models used, and verification relates to determining if the final tool can correctly produce results consistent with the model it is based upon. \cite{Tinsley2003}

Existing methods for surgical simulation have heavily relied on techniques for simulating soft body dynamics have typically been explored using total Lagrangian explicit dynamics (TLED) which have produced result  

\subsection{Virtual Reality}

One of the challenges in producing detailed virtual simulations is creating robust interfaces for setup and interaction Virtual reality has been considered for surgical simulation for several years and produced a variety of successful results as technology advances. In the area of surgical simulation, virtual reality primarily involves two major computational areas that are the simulating a surgical substrate that must be virtual but realistic, and providing a means by which an operator can interact reciprocally with the system. \cite{Spicer2004}

%Simplified virtual reality:
%Augmented virtual reality:
In this proposal virtual reality will refer to immersive virtual reality whereby the virtual environment aims to provide maximal primary sensory input/output. These systems may be supported by the use of props.