% Timeline
% Provide an estimated timeline of the steps needed to complete the work.
% Marking
% The student has reasonably set out the steps needed for the proposed method and timeline for completion given the student's experience in the area of study.
% Mark/10, Weight = 1

%April-May-June-July-August-September-October-November

%Wk9 => S1W1
%Wk17 => S1W8
%Wk10 => S1

%Proposal 20th April
%Progress Report: 4th -> 22nd May
%Abtract S2 Week7
%Seminar S2 9&10
%Final Report S2 Wk12

The timeline for this project will be set from April 20 to October 18. The scope of work for this project will involve five development epics (see \cref{section:methods}) in conjunction with five research deliverables. Each week of the project has an allocation of 10 working hours.

To avoid the issues related to waterfall development processes, each development epic has been given a total estimate (in hours) to allow for agile software development methodologies. This has the benefit of allowing for feedback, overlap and flexibility in terms of exact start and complete date of each epic by utilizing demo feedback (see \cref{section:demos})

The following deliverables will be prepared to allow for tracking of the project and ensuring objectives are being met.

\subsection{Progress Report}

An oral progress report will be prepared covering important information regarding the acquired resources, project architecture, ethics approval strategy, conceptual images of proposed demos, reporting of any discovered limitations of existing software and timeline adjustments.

\subsection{Demos}
\label{section:demos}

Demonstrations during development will be made and presented to gain early feedback and identify shortcomings every four weeks as software objectives are completed. Demonstrations will prepared in agile manor by enhancing prepared demonstration scenes as features are added. Demo scenes will include presenting simple geometry, brain geometry with segmentations and respective time series interventions for each.

\subsection{Abstract and Seminar}

A formal abstract and seminar will be prepared near the end of development to introduce the project and present methodologies, results and demonstrations in an easily comprehensible manor to an audience.

\subsection{Report}

A formal research report will be completed at the end of the project providing background on surgical simulation, discussion on the design process used, a high level overview of the final design and results, followed by discussion of future work and conclusions. 

%Build and install Slicer Virtual Reality from Source

%Compose a python script that binds basic VR input mappings to printing text.
%This may involve repository code modifications.
%Should contain for both hands: pointing, grabbing, joystick x/y, joystick click, button 1, button 2

%Unbind transforming (or at least the scaling at least component) from the regular grab action

%Add a raycasting from a pointing hand. Print pointed object name.

%Add highlighting of segmentation from hand raycasting.


